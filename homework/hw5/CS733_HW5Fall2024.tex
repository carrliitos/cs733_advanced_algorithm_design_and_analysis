\documentclass[12pt]{article}
%\pagestyle{empty} 
\usepackage{amsmath}
\usepackage{algpseudocode}
\usepackage{algorithm}
\usepackage[english]{babel}
\usepackage{amsthm}
\usepackage{amssymb}
\usepackage{color}
\usepackage{graphicx}
\usepackage{epsfig}

\newcommand{\vs}{\vspace{2mm}}
\newcommand{\ls}{\vspace{5mm}} 

\newcommand{\ms}{\vspace{3mm}}
\newcommand{\bc}{\begin{center}}
\newcommand{\ec}{\end{center}}
\newcommand{\sm}{\small}
\newcommand{\hs}{\hspace{10mm}}
\newcommand{\ha}{\hspace{1mm}}
\newcommand{\bo}{\rule{2mm}{3mm}}
\textheight=680pt
\textwidth=460pt
\hoffset=-50pt
\voffset=-50pt
%\topmargin=-0.5in
%\textheight=10in
%\oddsidemargin=0.125in
%\evensidemargin=0.125in
%\textwidth=7.5in
\begin{document}
\bc\ 
 { \bf Homework 5 (50 points) }  Due: October 11, 2024 11:59 pm\\
 { \bf COMPSCI 733: Advanced Algorithms and Designs } \\ 
\ec\
\ls\
\noindent{\bf Documentation:} (5 points)
Type your solutions using Latex \\
(www.overleaf.com or https://www.latex-project.org/ ). Submit your solutions (pdf is enough)  to Canvas. 

\vs\

\noindent{\bf Problem 1: } (15 points)
\vs\

Prove that if an edge $(u,v)$ is contained in some minimum spanning tree, then it is
a light edge crossing some cut of the graph.
\vs\

\begin{proof}   
Your proof goes here.
\end{proof}

\noindent{\bf Problem 2: } (15 points)
\vs\

Let  $G = (V,E,W)$ is a weighted connected (undirected) graph where all edges have distinct weights.Use proof by contradiction to prove that the Minimum Spanning Tree of $G$ is unique. 

\begin{proof} (by contradiction)
Suppose there are two different MSTs, $T_1$ and $T_2$. 

Complete the proof by getting a contradiction.
\end{proof}


\noindent{\bf Problem 3: } (15 points)
\vs\

Let  $G = (V,E,W)$ is a weighted connected (undirected) graph where all edges have distinct weights.Use induction on $|V|=n$, to prove that the Minimum Spanning Tree of $G$ is unique for all $n \geq 1$. 

\begin{proof} (by induction on $n$)

    \begin{itemize}
    \item Prove the base case.
    \item Write the inductive hypothesis ($n=k$).
    \item Prove the inductive step. (Hint: Start with any tree with $n= k+1$ vertices.)
    \item Conclusion. 
    \end{itemize}

\end{proof}

\end{document}
