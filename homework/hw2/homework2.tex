

\documentclass[12pt]{article}

%\pagestyle{empty} 
\usepackage{amsmath}
\usepackage{algpseudocode}
\usepackage{algorithm}
\usepackage[english]{babel}
\usepackage{amsthm}
\usepackage{amssymb}
\usepackage{color}
\usepackage{graphicx}
\usepackage{epsfig}

\newcommand{\vs}{\vspace{2mm}}
\newcommand{\ls}{\vspace{5mm}} 

\newcommand{\ms}{\vspace{3mm}}
\newcommand{\bc}{\begin{center}}
\newcommand{\ec}{\end{center}}
\newcommand{\sm}{\small}
\newcommand{\hs}{\hspace{10mm}}
\newcommand{\ha}{\hspace{1mm}}
\newcommand{\bo}{\rule{2mm}{3mm}}
\textheight=680pt
\textwidth=460pt
\hoffset=-50pt
\voffset=-50pt
%\topmargin=-0.5in
%\textheight=10in
%\oddsidemargin=0.125in
%\evensidemargin=0.125in
%\textwidth=7.5in
\begin{document}
\bc\ 
 { \bf Homework 2 (50 points)} Due: September 20, 2024 11:59 pm\\
 { \bf COMPSCI 733: Advanced Algorithms and Designs } \\ 
 { \bf Benzon Carlitos Salazar (salazarbc24@uww.edu) } 
\ec\
\ls\

\noindent{\bf Documentation:} (5 points)
Type your solutions using Latex \\
(www.overleaf.com or https://www.latex-project.org/ ). Submit your solutions (pdf is enough)  to Canvas. 




\vs\

\noindent{\bf Problem 1: (15 points)}
Assume we have a one dimensional array of real numbers with indices, $1, 2, \ldots, n$. The following pseudocodes for MERGE and MERGE-SORT are from CLRS textbook.


\begin{algorithm}
\caption{MERGE(A,p,q,r)}
\begin{algorithmic}[1]
\State{$n_1=q-p+1$}
\State{$n_2=r-q$}
\State{let $L[1\ldots n_1+1]$ and $R[1\ldots n_2+1]$  be new arrays}
\For {$i=1$ to $n_1$}
\State{$L[i]=A[p+i-1]$}
\EndFor
\For {$j=1$ to $n_2$}
\State{$R[j]=A[q+j]$}
\EndFor
\State{$L[n_1+1]= \infty$}
\State{$R[n_2+1]= \infty$}
\State{$i=1$}
\State{$j=1$}
\For {$k=p$ to $r$}
\If{$L[i] \leq R[j]$}
\State{$A[k]=L[i]$}
\State{$i=i+1$}
\Else
\State{$A[k]=R[j]$}
\State{$j=j+1$}
\EndIf
\EndFor
\end{algorithmic}
\end{algorithm}

\begin{algorithm}
\caption{MERGE-SORT(A,p,r)}
\begin{algorithmic}[1]
\If{$p < r$}
\State{$q= (p+r)/2$}
\State{MERGE-SORT$(A,p,q)$}
\State{MERGE-SORT$(A,q+1,r)$}
\State{MERGE$(A,p,q,r)$}
\EndIf
\end{algorithmic}
\end{algorithm}

\begin{itemize}
\item[(a)] Write pseudocodes with new parameters to modify the above MERGE and MERGE-SORT  that divide the array into three equal parts, sort them, and do a three-way merge as follows. 

Use a new parameter $s$ and write 
   $MERGE-3(A, p, q, r, s)$ and $MERGE-SORT-3(A, p, s)$, where $A$ is an array and $p, q, r,$ and $s$ are indices into the array such that $p \leq q \leq r < s.$ $MERGE$ function assumes that the subarrays $A[p, \ldots, q], A[q+1, \ldots, r]$ and $A[r+1, \ldots, s]$ are in sorted order.
Make sure to write the complete pseudo codes for the modified functions.


\begin{algorithm}
\caption{MERGE-3(A,p,q,r,s)}
\begin{algorithmic}[1]
\State{$n_1 = q - p + 1$}
\State{$n_2 = r - q$}
\State{$n_3 = s - r$}
\State{Let $L[1 \dots n_1+1]$, $M[1 \dots n_2+1]$, and $R[1 \dots n_3+1]$ be new arrays}

\For{$i = 1$ to $n_1$}
    \State{$L[i] = A[p+i-1]$}
\EndFor
\For{$j = 1$ to $n_2$}
    \State{$M[j] = A[q+j]$}
\EndFor
\For{$k = 1$ to $n_3$}
    \State{$R[k] = A[r+k]$}
\EndFor

\State{$L[n_1+1] = \infty$} \Comment{Sentinel values}
\State{$M[n_2+1] = \infty$}
\State{$R[n_3+1] = \infty$}

\State{$i = 1$}
\State{$j = 1$}
\State{$k = 1$}

\For{$l = p$ to $s$}
    \If{$L[i] \leq M[j]$ and $L[i] \leq R[k]$}
        \State{$A[l] = L[i]$}
        \State{$i = i + 1$}
    \ElsIf{$M[j] \leq L[i]$ and $M[j] \leq R[k]$}
        \State{$A[l] = M[j]$}
        \State{$j = j + 1$}
    \Else
        \State{$A[l] = R[k]$}
        \State{$k = k + 1$}
    \EndIf
\EndFor
\end{algorithmic}
\end{algorithm}
\begin{algorithm}

\caption{MERGE-SORT-3(A,p,s)}
\begin{algorithmic}[1]
\If{$p < s$}
    \State{$q = p + \left\lfloor \frac{s-p+1}{3} \right\rfloor - 1$} \Comment{First third}
    \State{$r = q + \left\lfloor \frac{s-p+1}{3} \right\rfloor$} \Comment{Second third}
    
    \State{MERGE-SORT-3$(A, p, q)$}
    \State{MERGE-SORT-3$(A, q+1, r)$}
    \State{MERGE-SORT-3$(A, r+1, s)$}
    
    \State{MERGE-3$(A, p, q, r, s)$}
\EndIf
\end{algorithmic}
\end{algorithm}

\item[(b)] Let $T(n)$ be the running time of MERGE-SORT-3 on an array of size $n$. Write a recurrence relation (i.e., $T(n) = ? )$ for your algorithm. \\

Let \( T(n) \) be the running time of the MERGE-SORT-3 algorithm. The recurrence relation for \( T(n) \) is as follows:

\[ T(n) = 3T\left(\frac{n}{3}\right) + cn \quad \text{for} \, n > 1 \]

where \( c \) is a constant representing the time complexity of the merge step.

The base case occurs when the size of the array is 1:

\[ T(1) = O(1) \]

\end{itemize}
\vs\

In Problem 2 and 3, you need to complete the details  of a proof for the correctness of Merge Sort. You need to refer to the above Algorithm 2, $MERGE\_SORT(A,P,r)$,  and Algorithm 1, $MERGE(A,p,q,r)$, given in the CLRS textbook. We will establish correctness of Merge-sort in two parts:

\begin{itemize}
    
\item[1.] Assuming correctness of the Merge procedure, prove the correctness of MergeSort.
\item[2.] Prove the correctness of the  Merge procedure.

\end{itemize}
\noindent{\bf Problem 2: }(15 points)

Prove: Assuming that the procedure for Merge is correct, a call to $Merge\_Sort(A, p, r)$,
$p \leq r$, returns the elements in $A[p..r]$ rearranged in sorted order, and does not alter any entry
outside of the subarray $A[p..r].$

\begin{proof}:
We prove the correctness of the procedure \textsc{Merge-Sort} by strong induction on $m = r - p + 1$, i.e., by induction on the size of the subarray $A[p..r]$. \\

\textbf{Base case:} When $p = r$, i.e., $m = 1$, the subarray $A[p..r]$ contains only one element. Since a single element is trivially sorted, \textsc{Merge-Sort} correctly sorts the subarray. \\

\textbf{Induction hypothesis:} Assume that \textsc{Merge-Sort} correctly sorts any subarray $A[p..r]$ of size $1 \leq m < k$. That is, \textsc{Merge-Sort} correctly sorts any subarray with fewer than $k$ elements. \\

\textbf{Induction step:} We must prove that \textsc{Merge-Sort} correctly sorts the subarray $A[p..r]$ when $m = k$. \\

Consider the procedure \textsc{Merge-Sort}(A, p, r):

\begin{itemize}
    \item[1. ] If $p < r$, the algorithm computes the midpoint $q = \lfloor (p + r) / 2 \rfloor$. This divides the array into two subarrays: $A[p..q]$ and $A[q+1..r]$.
    \item[2. ] The algorithm recursively calls \textsc{Merge-Sort}(A, p, q) to sort the left half and \textsc{Merge-Sort}(A, q+1, r) to sort the right half.
    
    By the induction hypothesis, \textsc{Merge-Sort}(A, p, q) correctly sorts the left subarray and \textsc{Merge-Sort}(A, q+1, r) correctly sorts the right subarray.

    \item[3. ] After the recursive calls, the two subarrays are sorted, and the \textsc{Merge}(A, p, q, r) procedure is called to merge them into a single sorted subarray $A[p..r]$.

    Since we assume that \textsc{Merge} is correct, it correctly merges the two sorted subarrays into one sorted array.

    Additionally, \textsc{Merge-Sort} only modifies the elements in $A[p..r]$ and does not alter any entries outside this subarray.

    Thus, \textsc{Merge-Sort} correctly sorts $A[p..r]$ and does not modify any elements outside of $A[p..r]$ when $m = k$.
\end{itemize}

By the principle of strong induction, we conclude that \textsc{Merge-Sort} correctly sorts any subarray $A[p..r]$ and does not alter any entries outside of this subarray.
\end{proof}
\vs\
\pagebreak

\noindent{\bf Problem 3:} (15 points)
Correctness of Merge.

Merge procedure copies subarray $A[p..q]$ into $L[1..n1]$ and $A[q+1..n2]$ into $R[1..n2]$, with $L[n1+1]$
and $R[n2 + 1]$ set to $\infty$ (a value larger than any of the elements in $A[p..r]$).

Correctness of Merge is established through the correctness of the following loop
invariant for the for loop in Line 14 of the above Algorithm 1:

{\bf Loop invariant: } 

At the start of each iteration of the for loop in line 14, $A[p..k -1]$ contains the $k-p$ smallest
elements in $L[1..n1 + 1]$ and $R[1..n2 + 1]$ in sorted order. Further, $L[i]$ and $R[j]$ are the
smallest elements in their arrays that have not been copied back into $A.$ The elements in
array A outside of subarray $A[p..r]$ are unchanged.

{\bf Complete the following steps:}
\begin{itemize}
\item[1.]   Show Initialization holds.This establishes the base case by proving that the loop invariant holds just
before the start of the first iteration.
\item[2.]Show Maintenance  holds.Assuming that the loop invariant holds at the start of a given iteration this
establishes that it continues to hold at the start of the next iteration.

\item[3.]   Show Termination holds.States what the loop invariant establishes about the computation at the time
when the loop is exited.

\begin{proof}
We prove the correctness of the procedure \textsc{Merge} by induction on the size of the subarrays being merged.

\textbf{Base case:} Consider the smallest size for the subarrays, when each subarray contains just one element, i.e., when $q = p$ and $r = q+1$. In this case, we are merging two single-element subarrays $A[p]$ and $A[q+1]$.

Since each subarray contains only one element, they are already sorted. The \textsc{Merge} procedure compares $A[p]$ and $A[q+1]$, placing the smaller element in the merged subarray $A[p..r]$.

Thus, the merged subarray $A[p..r]$ is sorted, and no elements outside this subarray are altered. The base case holds.

\textbf{Induction hypothesis:} Assume that the \textsc{Merge} procedure correctly merges any two sorted subarrays of size less than $n$. That is, for arrays of size $r - p + 1 < n$, \textsc{Merge} correctly merges the subarrays $A[p..q]$ and $A[q+1..r]$ into a sorted array $A[p..r]$.

\textbf{Induction step:} We now prove that \textsc{Merge}(A, p, q, r) correctly merges the sorted subarrays $A[p..q]$ and $A[q+1..r]$ when $n = r - p + 1$.

The procedure \textsc{Merge} creates two auxiliary arrays $L[1..n_1+1]$ and $R[1..n_2+1]$, where $n_1 = q - p + 1$ and $n_2 = r - q$. The array $L$ stores the elements of the left subarray $A[p..q]$, and $R$ stores the elements of the right subarray $A[q+1..r]$. Sentinel values (infinity) are added to the ends of both $L$ and $R$ to simplify the comparison process.

The procedure then merges $L$ and $R$ into $A[p..r]$. It uses two pointers, $i$ and $j$, initialized to 1, to point to the current element in $L$ and $R$, respectively. A loop runs from $k = p$ to $r$:
\begin{itemize}
    \item If $L[i] \leq R[j]$, then $A[k] = L[i]$, and $i$ is incremented.
    \item Otherwise, $A[k] = R[j]$, and $j$ is incremented.
\end{itemize}

Since both $L$ and $R$ are sorted, each comparison ensures that the smallest remaining element from either $L$ or $R$ is placed into $A[k]$. After the loop completes, the subarray $A[p..r]$ is fully sorted, and no elements outside the range $A[p..r]$ are altered.

By the induction hypothesis, \textsc{Merge} correctly merges the sorted subarrays $A[p..q]$ and $A[q+1..r]$ into a single sorted subarray $A[p..r]$.

Thus, by induction, we conclude that the \textsc{Merge} procedure correctly merges two sorted subarrays into a single sorted subarray and does not alter any entries outside the specified range.
\end{proof}
\end{itemize}
\end{document}
