

\documentclass[12pt]{article}

%\pagestyle{empty} 
\usepackage{amsmath}
\usepackage{algpseudocode}
\usepackage{algorithm}
\usepackage[english]{babel}
\usepackage{amsthm}
\usepackage{amssymb}
\usepackage{color}
\usepackage{graphicx}
\usepackage{epsfig}

\newcommand{\vs}{\vspace{2mm}}
\newcommand{\ls}{\vspace{5mm}} 

\newcommand{\ms}{\vspace{3mm}}
\newcommand{\bc}{\begin{center}}
\newcommand{\ec}{\end{center}}
\newcommand{\sm}{\small}
\newcommand{\hs}{\hspace{10mm}}
\newcommand{\ha}{\hspace{1mm}}
\newcommand{\bo}{\rule{2mm}{3mm}}
\textheight=680pt
\textwidth=460pt
\hoffset=-50pt
\voffset=-50pt
%\topmargin=-0.5in
%\textheight=10in
%\oddsidemargin=0.125in
%\evensidemargin=0.125in
%\textwidth=7.5in
\begin{document}
\bc\ 
 { \bf  Homework  1 (50 points)}  Due: September 13, 2024 11:59 pm\\
 { \bf COMPSCI 733: Advanced Algorithms and Designs } \\ 
 { \bf Benzon Carlitos Salazar (salazarbc24@uww.edu) } 
\ec\
\ls\

\noindent{\bf Documentation:} (5 points)
Type your solutions using Latex \\
(www.overleaf.com or https://www.latex-project.org/ ). Submit your solutions (pdf is enough)  to Canvas. 




\vs\

\noindent{\bf Problem 1: (15 points)}
Assume we have a one dimensional array of real numbers with indices, $1, 2, \ldots, n$. The following pseudocodes for MERGE and MERGE-SORT are from CLRS textbook.


\begin{algorithm}
\caption{MERGE(A,p,q,r)}
\begin{algorithmic}[1]
\State{$n_1=q-p+1$}
\State{$n_2=r-q$}
\State{let $L[1\ldots n_1+1]$ and $R[1\ldots n_2+1]$  be new arrays}
\For {$i=1$ to $n_1$}
\State{$L[i]=A[p+i-1]$}
\EndFor
\For {$j=1$ to $n_2$}
\State{$R[j]=A[q+j]$}
\EndFor
\State{$L[n_1+1]= \infty$}
\State{$R[n_2+1]= \infty$}
\State{$i=1$}
\State{$j=1$}
\For {$k=p$ to $r$}
\If{$L[i] \leq R[j]$}
\State{$A[k]=L[i]$}
\State{$i=i+1$}
\Else
\State{$A[k]=R[j]$}
\State{$j=j+1$}
\EndIf
\EndFor
\end{algorithmic}
\end{algorithm}

\begin{algorithm}
\caption{MERGE-SORT(A,p,r)}
\begin{algorithmic}[1]
\If{$p < r$}
\State{$q= (p+r)/2$}
\State{MERGE-SORT$(A,p,q)$}
\State{MERGE-SORT$(A,q+1,r)$}
\State{MERGE$(A,p,q,r)$}
\EndIf
\end{algorithmic}
\end{algorithm}

\begin{itemize}
\item[(a)] Write pseudocodes with new parameters to modify the above MERGE and MERGE-SORT  that divide the array into three equal parts, sort them, and do a three-way merge as follows. 

Use a new parameter $s$ and write 
   $MERGE-3(A, p, q, r, s)$ and $MERGE-SORT-3(A, p, s)$, where $A$ is an array and $p, q, r,$ and $s$ are indices into the array such that $p \leq q \leq r < s.$ $MERGE$ function assumes that the subarrays $A[p, \ldots, q], A[q+1, \ldots, r]$ and $A[r+1, \ldots, s]$ are in sorted order.
Make sure to write the complete pseudo codes for the modified functions.


\begin{algorithm}
\caption{MERGE-3(A,p,q,r,s)}
\begin{algorithmic}[1]
\State{Your pseudocode goes here.}
\end{algorithmic}
\end{algorithm}
\begin{algorithm}

\caption{MERGE-SORT-3(A,p,s)}
\begin{algorithmic}[1]
\State{Your pseudocode goes here.}
\end{algorithmic}
\end{algorithm}

\item[(b)] Let $T(n)$ be the running time of MERGE-SORT-3 on an array of size $n$. Write a recurrence relation (i.e., $T(n) = ? )$ for your algorithm. 
\end{itemize}
\vs\



In Problem 2 and 3, you need to complete the details  of a proof for the correctness of Merge Sort. You need to refer to the above Algorithm 2, $MERGE\_SORT(A,P,r)$,  and Algorithm 1, $MERGE(A,p,q,r)$, given in the CLRS textbook. We will establish correctness of Merge-sort in two parts:

\begin{itemize}
    
\item[1.] Assuming correctness of the Merge procedure, prove the correctness of MergeSort.
\item[2.] Prove the correctness of the  Merge procedure.

\end{itemize}
\noindent{\bf Problem 2: }(15 points)

Prove: Assuming that the procedure for Merge is correct, a call to $Merge\_Sort(A, p, r)$,
$p \leq r$, returns the elements in $A[p..r]$ rearranged in sorted order, and does not alter any entry
outside of the subarray $A[p..r].$

\begin{proof}:  By strong induction on $m = r - p + 1$ (i.e., by induction on the size of the
sub-array $A[p..r]$).

Base case: Need to show correctness for the smallest input size. This occurs when $p = r,$ i.e.,
when $m = 1.$ Since the
sub-array $A[p..r]$) has only one element, it is trivially sorted.

Induction step. Assume correctness for all sizes $1 \leq k < m$, and establish correctness when $k= m$. {\bf Fill in the details of this induction step.}
\end{proof}
\vs\
\pagebreak

\noindent{\bf Problem 3:} (15 points)
Correctness of Merge.

Merge procedure copies subarray $A[p..q]$ into $L[1..n1]$ and $A[q+1..n2]$ into $R[1..n2]$, with $L[n1+1]$
and $R[n2 + 1]$ set to $\infty$ (a value larger than any of the elements in $A[p..r]$).

Correctness of Merge is established through the correctness of the following loop
invariant for the for loop in Line 14 of the above Algorithm 1:

{\bf Loop invariant: } 

At the start of each iteration of the for loop in line 14, $A[p..k -1]$ contains the $k-p$ smallest
elements in $L[1..n1 + 1]$ and $R[1..n2 + 1]$ in sorted order. Further, $L[i]$ and $R[j]$ are the
smallest elements in their arrays that have not been copied back into $A.$ The elements in
array A outside of subarray $A[p..r]$ are unchanged.

{\bf Complete the following steps:}
\begin{itemize}
\item[1.]	Show Initialization holds.This establishes the base case by proving that the loop invariant holds just
before the start of the first iteration.
\item[2.]Show Maintenance  holds.Assuming that the loop invariant holds at the start of a given iteration this
establishes that it continues to hold at the start of the next iteration.

\item[3.]	Show Termination holds.States what the loop invariant establishes about the computation at the time
when the loop is exited.
\end{itemize}


\end{document}




